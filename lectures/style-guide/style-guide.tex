\documentclass{coq311notes}

\usepackage{coq311}

\setnotetitle{Coq Style Guide}
\setnoteimage{../img/Coq}

\newcommand{\todo}[1]{{\bf TODO: } #1}

\begin{document}
\maketitle

\begin{quote}
  Everything should be as simple as possible, but no simpler.\\
  \-- Albert Einstein
\end{quote}

Like in natural language, the use of a consistent idiomatic vernacular
distinguishes the novice from the expert. In this course one of the
goals is to move beyond merely requiring program correctness, and
focus on program \emph{elegance}. This requires writing programs in an
idiomatic style that emphasizes clarity, brevity, and elegance.

If you come from a background in other programming languages such as
C, Lisp or Python you will promptly discover that programming in Coq
is quite different. The goal of this document is to point out some of
the particular idiosyncracies of Coq that will get you writing
idiomatic code more quickly.

\section{}

\end{document}

\documentclass{article}

%% \usepackage{amsmath}
\usepackage{hyperref}
\usepackage{palatino}
\usepackage[margin=1in]{geometry}

\renewcommand{\maketitle}{{\flushleft  \textsc{Course Overview}\\\vspace{0.1cm}\hrule\vspace{0.2cm}}}


\begin{document}
\maketitle
So far we have written a few small programs.
How do we know these programs are correct?
Our arguments have been extremely casual.

In these notes we introduce unit testing, explain the syntax we will use in this course, and explore commonly-used unit testing strategies.

\subsection*{Informal Correctness}
Last time, we introduced a type for natural numbers and defined a few functions on them.
Two of the functions we wrote were successor and \texttt{plus}: the first added 1 to a number and the second took the sum of two numbers.
\begin{verbatim}
Definition add1 (n : nat) : nat := S n.
\end{verbatim}
\begin{verbatim}
Fixpoint plus (n1 : nat) (n2 : nat) : nat :=
  match n1 with
  | O   => n2
  | S n => add1 (plus n n2)
  end.
\end{verbatim}
The function \texttt{add1} is pretty obviously correct.
We defined natural numbers as either zero or the successor of another number, so it makes sense that we can add 1 by taking the successor.
Determining that \texttt{plus} is correct is slightly more difficult, but inspecting the code shows that we are just counting; we repeatedly add 1 to the number \texttt{n2}.

This informal approach of inspecting code does not scale well.
Here is a function BLAH.
Can you figure out what it does?
If you like, try running the code.
\begin{verbatim}
(* insert complex function here *)
\end{verbatim}

\coffeebreak{How did you go about understanding BLAH? In general, what are good strategies for learning about a new piece of code?}

Observations/assertions
\begin{itemize}
\item
  ``Obviously correct'' falls apart very quickly.
\item
  Learning by example is often very effective.
\end{itemize}

\subsection*{Unit Testing}
A unit test is XXX.

We write unit tests in Coq311 by XXX.

\coffeebreak{Think of some tests for \texttt{plus}. Are you certain that it works correctly for any two natural numbers?}

\subsection*{Beyond Testing}
Proving!
(Just a taste for now)

\subsection*{Lit. Review}
Unit testing is everywhere.
(Insert quote by Tony Hoare.)

\subsubsection*{Testing Strategies}
\begin{itemize}
\item Exhaustive
\item Path-Complete
\item Edge Testing
\item Randomized
\item Black Box
\item Glass Box
\end{itemize}

\subsubsection*{Testing Frameworks}
There are many famous tools for helping programmers create, write, and run unit tests.
\begin{itemize}
\item JUnit
\item QuickCheck
\item Fuzz-Testing tools, something in PLT
\end{itemize}

\subsubsection*{Other Stuff}
Books on testing, wild stuff like POPL'14 test-driven-synthesis.

\end{document}

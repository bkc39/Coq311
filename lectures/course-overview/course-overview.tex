\documentclass{article}

%% \usepackage{amsmath}
\usepackage{hyperref}
\usepackage{palatino}
\usepackage[margin=1in]{geometry}

\renewcommand{\maketitle}{{\flushleft  \textsc{Course Overview}\\\vspace{0.1cm}\hrule\vspace{0.2cm}}}


\begin{document}
\maketitle

\begin{quote}
  ``A computer is simply an instrument whose music is ideas'' -Alan Kay, 2003\footnote{\url{http://www.scholastic.com/browse/article.jsp?id=5}}
\end{quote}

The goal of this course is the explain and support this quotation.
We will teach fundamental design and programming skills using the Coq proof assistant.

\subsection*{Target Audience}
This course is intended for students with at least one year of programming experience.
You should have some practice building small applications.
Experience in any language is fine, but a strong background in functional programming will subsume most material in the first third of the course.
Experience with an imperative language like Java, C, and Python is good, but we recommend looking over the course prologue~\cite{TODO} to un-learn a few things before starting the first lecture.

\subsection*{Topics Covered}
TBA

\subsection*{Topics not Covered}
Subject to change, but probably including:

\begin{center}
  \begin{tabular}{c c c c}
    Amortized Analysis & Complexity & Concurrency & Coinduction
  \\Data-Oriented Programming & & &
  \end{tabular}
\end{center}

\subsection*{Structure}
Our general problem-solving strategy will be:
\begin{enumerate}
\item Think about the essence of the problem, choose a datatype to model it.
\item Convince ourselves that we can solve the problem with psuedocode and an informal correctness proof.
\item Create a program realizing our ideas.
\item Create a formal proof of our program's correctness.
\end{enumerate}

Steps 1 and 2 are by far the most important and broadly applicable; however ideas are worth little until they can be implemented.
The order of steps 3 and 4 is guided by our choice of language.
There is a strong connection between formally proving a theorem and programming an algorithm, and these activities are best done simultaneously, but the ``prove, then program'' order seems to work best in Coq.

\subsection*{About Coq}
We will use the Coq programming language / proof assistant because it offers an expressive language for specifying problems and reasoning about correctness.
We can uniformly express a data model, algorithm, and proof of correctness in Coq.

\begin{center}
\begin{verbatim}
  Inductive data := ...
\end{verbatim}
\begin{verbatim}
  Fixpoint program : data -> data := ...
\end{verbatim}
\begin{verbatim}
  Theorem about_the_program : forall A, exists B ... 
  Proof. ...
\end{verbatim}
\end{center}

Furthermore, Coq interactively evaluates our definitions for basic syntactical and logical errors.
That said, Coq has problems.
Most of the above-listed ``topics not covered'' are omitted because Coq is not mature enough to handle them.
Despite its shortcomings and learning curve, we think Coq is a great tool for learning and well worth knowing.

\subsection*{Conventions}

The following are idiosyncrasies we use throughout the text.

\subsubsection*{Coffee Breaks}

We will sometimes write a small question on its own line, centered, and surrounded by flowers, like this:

\coffeebreak{Did you know that the Verranzano Bridge is 1,439 meters longer than the Golden Gate Bridge?}

These questions will be relevant to the text, and nearly always answered by the following paragraph.
We include them as an invitation to stand up, stretch, pour a cup of coffee, or gaze out a window.
Take a two-minute break and think about the question before going forward.

\end{document}

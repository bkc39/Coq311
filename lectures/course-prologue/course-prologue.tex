\documentclass{article}

%% \usepackage{amsmath}
\usepackage{hyperref}
\usepackage{palatino}
\usepackage[margin=1in]{geometry}

\renewcommand{\maketitle}{{\flushleft  \textsc{Course Overview}\\\vspace{0.1cm}\hrule\vspace{0.2cm}}}


\begin{document}
\maketitle

Computer programming is more than the act of giving instructions to a computer.
Computer programming is the art of expressing complex ideas unambiguously and clearly.
All code you write should be written first for human readers; the fact that a computer will later interpret the bits is a secondary concern.

\subsection*{An Example}
Suppose you are writing a computer program and want to know the value of 4!.
One possible answer is the integer 24.
You can use the literally type the constant \texttt{24} and use it forever more as the value of 4!.

Another correct answer is \texttt{n * 4}.
This is supposing you have a variable \texttt{n} saved in your current programming environment that has the value 3! stored to it.

The answer we prefer is \texttt{factorial(4)}, where \texttt{factorial : nat -> nat} is a stateless (read: functional) subroutine specifying how to compute the factorial of any number.
Like the previous answer, this assumes we have a special variable in our environment (\texttt{factorial} in this case, \texttt{n} previously).
However, the factorial function is worth finding or implementing because we can re-use it.

\subsection*{Variables}
A variable is not a pointer to a register whose value changes over time.
A variable is an unknown quantity.
To quote Professor Robert Harper, ``you had it right the first time'' when you learned variables in algebra.

\end{document}
